\section{国内的研究现状}

\textcolor{red}{
潮汐、海流和波浪研究
1956年,海军编制出中国一些主要港口的潮汐表;青岛验潮站的平均海平面被确定为中国陆地高程基准面。60年代初,编制出渤海、黄海、东海、南海及舟山群岛附近海区、北部湾等 6个海区的永久潮流表和半日潮流图。从1980年起,国家海洋局开始编制世界潮汐表。在潮汐理论研究方面,1964年完成了《中国近海的潮波系统》研究报告。以后,对陆架海的潮汐理论进行研究,包括摩擦的非线性效应、摩擦对海湾中潮波的影响、黄海潮能消耗等课题。还开展了中国古潮汐理论的研究。
中国学者比较系统地研究了中国近海海流系统的结构、途径、性质、强度、变化、沿岸流与外海流的相互关系,黑潮流速与地形关系等问题。总结了东海海流系的基本流型和变化,提出东海海流系统的基本模式;发现了南海表层流受季风控制,具有漂流性质,指出冬季存在一个逆向的“南海暖流”;提出了余流、地转流和考虑涡动、摩擦效应的海流计算方法。
文圣常等从50年代起,应用能量平衡和谱方法结合起来的观点,提出了一种海浪预报方法,并对浅海风浪形成、涌浪传播和近岸波变化等进行研究。一些学者还结合海港工程、石油钻探设施、船舶设计等,对波浪理论的应用进行了研究,为工程提供最优设计参数。
}

\textcolor{red}{
海洋观测仪器方面
中国从50年代开始研制和生产海洋常规观测仪器。60年代和70年代初,先后组织两次全国海洋仪器技术攻关,研制出各种海洋观测仪器50多种,包括金属弹簧重力仪、振弦式海洋重力仪和核子旋进海洋磁力仪。70年代末以来,中国的海洋仪器逐步向自记、走航、遥测、遥控方向发展。到1984年底,中国已研制和生产的海洋仪器达 130多种。水声技术、海洋遥感技术、激光技术、电子计算机在海洋上的应用技术等也都有了长足的进展
}
