\section{我是第一节简介}

\subsection{海浪的信息重要性}
mod : 2014年 09月 25日 星期四 07:55:10 CST\\

海洋, 作为地球上占据最大面积的地理形态,深深的影响着人类的生存环境。而海浪,作为海洋与陆地以及人类接触的最直接的媒介,时时刻刻的反映着海洋的信息。关于特定海域的海浪的物理信息,一直是人们希望得到,并加以研究的重要资源。

\subsection{对海浪信息的研究历史}
自古以来,人类在打渔,船运的过程中,积累了丰富的关于海浪的经验。我国也是世界上最早意识到需要认识,了解和利用海浪的国家之一。古人很早就已能从海浪信息中,获取“舟楫之便”\footnote{http://www.baike.com/wiki/中国海洋研究史}。史书记载,自三国时期,便出现了我国第一篇关于潮汐的专论──严畯的《潮水论》。到了唐宋时期,潮汐研究已取得了相当高的水平。1405~1433年,明朝郑和下西洋,最远到达赤道以南的非洲东海岸和马达加斯加岛。可见,在古时相当长的一段时间内,中国对海浪的认识和利用在世界上是居于前列的。

在古代,我国对海洋的研究主要集中在海洋潮汐、海洋气象、海洋地貌、和海洋生物4个方面。而本论文要研究的海浪的相关内容,是前三者的一个总合的产物。

\textcolor{red}{
在海事活动中,风是至关重要的天气要素,所以在古代对风的认识较为深刻。中国古代水手、渔民知道用各种方法预测海洋风暴。他们把一年中海上常有风暴的日期记下来,称为“暴日”或“飓日”。一些航海书籍中记有全年暴日及其名称,如《顺风相送》中有逐月恶风条。并总结出暴风季节发生的规律和暴日在不同时节的频率,从而找出海上活动的危险期和安全期。古代预测台风的一种办法是观察海洋现象。海洋长浪有很高的运动速度,台风还在外洋时,其形成的长浪已传播到近海,形成涌浪,造成潮汐异常、海底淤泥搅起、海水发臭、海洋动物表现异常等现象。人们把上述现象称之为“天神未动,海神先动”,并把这种无风的涌浪称为“移浪”或“风潮”。
中国很早就以风作动力,用帆助航。东汉时,利用季风航海已有文字记载,把每年梅雨后出现的东南季风称为“舶?风”。唐、宋以后,利用季风航海十分广泛。明代郑和7次出海, 多在冬、春季节利用东北季风启航,又多在夏、秋季节利用西南季风返航,说明他们已较充分地认识和利用了亚洲南部、北印度洋上风向和海流季节性变化的规律。在航行途中他们观察日月星辰的出没和位移、风向、天色、云状、霾雾、气温及洋面波涛的变化,预测海洋气象、水文潮汐的变化趋势,保证了航行的安全。
}



